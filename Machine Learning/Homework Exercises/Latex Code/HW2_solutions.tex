\documentclass[12pt]{book}

%These tell TeX which packages to use.
\usepackage{array,epsfig}
\usepackage{amsmath}
\usepackage{amsfonts}
\usepackage{amssymb}
\usepackage{amsxtra}
\usepackage{amsthm}
\usepackage{mathrsfs}
\usepackage{color}
\usepackage{enumitem}
\usepackage{bm}

%Here I define some theorem styles and shortcut commands for symbols I use often
\theoremstyle{definition}
\newtheorem{defn}{Definition}
\newtheorem{thm}{Theorem}
\newtheorem{cor}{Corollary}
\newtheorem*{rmk}{Remark}
\newtheorem{lem}{Lemma}
\newtheorem*{joke}{Joke}
\newtheorem{ex}{Example}
\newtheorem*{soln}{Solution}
\newtheorem{prop}{Proposition}

\newcommand{\lra}{\longrightarrow}
\newcommand{\ra}{\rightarrow}
\newcommand{\surj}{\twoheadrightarrow}
\newcommand{\graph}{\mathrm{graph}}
\newcommand{\bb}[1]{\mathbb{#1}}
\newcommand{\Z}{\bb{Z}}
\newcommand{\Q}{\bb{Q}}
\newcommand{\R}{\bb{R}}
\newcommand{\C}{\bb{C}}
\newcommand{\N}{\bb{N}}
\newcommand{\M}{\mathbf{M}}
\newcommand{\m}{\mathbf{m}}
\newcommand{\MM}{\mathscr{M}}
\newcommand{\HH}{\mathscr{H}}
\newcommand{\Om}{\Omega}
\newcommand{\Ho}{\in\HH(\Om)}
\newcommand{\bd}{\partial}
\newcommand{\del}{\partial}
\newcommand{\bardel}{\overline\partial}
\newcommand{\textdf}[1]{\textbf{\textsf{#1}}\index{#1}}
\newcommand{\img}{\mathrm{img}}
\newcommand{\ip}[2]{\left\langle{#1},{#2}\right\rangle}
\newcommand{\inter}[1]{\mathrm{int}{#1}}
\newcommand{\exter}[1]{\mathrm{ext}{#1}}
\newcommand{\cl}[1]{\mathrm{cl}{#1}}
\newcommand{\ds}{\displaystyle}
\newcommand{\vol}{\mathrm{vol}}
\newcommand{\cnt}{\mathrm{ct}}
\newcommand{\osc}{\mathrm{osc}}
\newcommand{\LL}{\mathbf{L}}
\newcommand{\UU}{\mathbf{U}}
\newcommand{\support}{\mathrm{support}}
\newcommand{\AND}{\;\wedge\;}
\newcommand{\OR}{\;\vee\;}
\newcommand{\Oset}{\varnothing}
\newcommand{\st}{\ni}
\newcommand{\wh}{\widehat}

%Pagination stuff.
\setlength{\topmargin}{-.3 in}
\setlength{\oddsidemargin}{0in}
\setlength{\evensidemargin}{0in}
\setlength{\textheight}{9.in}
\setlength{\textwidth}{6.5in}
\pagestyle{empty}



\begin{document}


\begin{center}
{\Large Machine Learning and Computational Statistics}\\
\large\textbf{Theodoros Georgiopoulos}\\ %You should put your name here
\Large Homework 2 %You should write the date here.
\end{center}

\vspace{0.2 cm}


\subsection*{Exercises for Unit 2: Classification and Statistics concepts}

\vspace{0.2 cm}

{\underline{\large Solution for exercise 3}}
\vspace{0.1 cm}

\noindent
Consider the following non linear model:
\begin{equation}
y = 3x_1^2 + 4x_2^2 + 5x_3^2 + 7x_1x_2 + x_1x_3 \\
+ 4x_2x_3 - 2 x_1 - 3x_2 - 5x_3 + \eta.
\end{equation}
The suitable function $\phi$ that transforms the problem to a space where the problem of estimating the model becomes linear is

\begin{equation}
\bm{\phi}(x) = \begin{bmatrix}
x_{1}^2  \\
x_2^2 \\
x_3^2 \\
x_1x_2 \\
x_1x_3\\
x_2x_3\\
x_1\\
x_2\\
x_3\\
\end{bmatrix}.
\end{equation}
With this transformation the $2$-dimensional space becomes $9$-dimensional space and the problem is now linear.

\vspace{0.5 cm}
\noindent
{\underline{\large Solution for exercise 4}}
\vspace{0.3 cm}

\noindent
For $\bm{x} = [x_1,x_2,x_3]^T$, consider the following non linear classification task:
\begin{equation}
x_1^2 + 3x_2^2 + 6x_3^2 + x_1x_2 + x_2x_3 > (<)3 \rightarrow \bm{x} \in \omega_1 (\omega_2).
\end{equation}
The suitable function $\phi$ that transforms the problem to a space where the problem
of estimating the border of the two classes becomes linear is

\begin{equation}
\bm{\phi}(x) = \begin{bmatrix}
x_{1}^2  \\
x_2^2 \\
x_3^2 \\
x_1x_2 \\
x_2x_3\\
\end{bmatrix}.
\end{equation}

\noindent
The dimension of the original space is $2$ and the dimension of the transformed space in which the problem is linear, is $5$.

\newpage

\vspace{0.5 cm}
\noindent
{\underline{\large Solution for exercise 6}}
\vspace{0.3 cm}

\noindent
Let $x, y$ be two random variables with sample spaces $X = \{x_1,x_2,\dots,x_{n_x}\}$ and $Y = \{y_1,y_2,\dots,y_{n_y}\}$ respectively. Moreover, we denote $n$ the total number of experiments and as $\{n_1^X,n_2^X,\dots,n_{n_x}^X\}$ the probabilities of occurence of $x_1,x_2,\dots,x_{n_x}$. Similarly, we denote as  $\{n_1^Y,n_2^Y,\dots,n_{n_y}^Y\}$ the probabilities of occurence of $y_1,y_2,\dots,y_{n_y}$. Finally, we denote as $n_{ij}$ the number of the times that the value $x_i$ and $y_j$ occured simultaneously. The joint probability can be approximated as $P(x,y) \approx \frac{n_{ij}}{n}$. The total number $n_i^X$ that value $x_i$ has occured is $n_i^X = \sum_{i=1}^{n_y}n_{ij}$. Dividing both sides by $n$ gives
\begin{equation}
P(x) = \sum_{y\in Y}P(x,y).
\end{equation}

\noindent
The conditional probability $P(A|B)$ is defined as
\begin{equation}
P(A|B) = \frac{n_{AB}}{n} \frac{n}{n_B} = \frac{n_{AB}}{n_B}.
\end{equation}
So, for the product rule, we can write
\begin{equation}
P(A|B)p(B) = \frac{n_{AB}}{n_B}\frac{n_B}{n} = \frac{n_{AB}}{n} = P(A,B).
\end{equation}
Using the product rule, we can verify the Bayes rule
\begin{equation}
P(A|B)p(B) = \frac{n_{AB}}{n_B}\frac{n_B}{n} = \frac{n_{AB}}{n_A}\frac{n_A}{n} = P(B|A)p(A).
\end{equation}
So,
\begin{equation}
P(B|A) = \frac{P(A|B)P(B)}{P(A)}.
\end{equation}
\end{document}


